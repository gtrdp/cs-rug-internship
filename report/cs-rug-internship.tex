\documentclass[journal]{vgtc}                % final (journal style)
%\documentclass[review,journal]{vgtc}         % review (journal style)
%\documentclass[widereview]{vgtc}             % wide-spaced review
%\documentclass[preprint,journal]{vgtc}       % preprint (journal style)

%% Uncomment one of the lines above depending on where your paper is
%% in the conference process. ``review'' and ``widereview'' are for review
%% submission, ``preprint'' is for pre-publication, and the final version
%% doesn't use a specific qualifier.

%% Please use one of the ``review'' options in combination with the
%% assigned online id (see below) ONLY if your paper uses a double blind
%% review process. Some conferences, like IEEE Vis and InfoVis, have NOT
%% in the past.

%% Please note that the use of figures other than the optional teaser is not permitted on the first page
%% of the journal version.  Figures should begin on the second page and be
%% in CMYK or Grey scale format, otherwise, colour shifting may occur
%% during the printing process.  Papers submitted with figures other than the optional teaser on the
%% first page will be refused.

%% These few lines make a distinction between latex and pdflatex calls and they
%% bring in essential packages for graphics and font handling.
%% Note that due to the \DeclareGraphicsExtensions{} call it is no longer necessary
%% to provide the the path and extension of a graphics file:
%% \includegraphics{diamondrule} is completely sufficient.
%%
\ifpdf%                                % if we use pdflatex
  \pdfoutput=1\relax                   % create PDFs from pdfLaTeX
  \pdfcompresslevel=9                  % PDF Compression
  \pdfoptionpdfminorversion=7          % create PDF 1.7
  \ExecuteOptions{pdftex}
  \usepackage{graphicx}                % allow us to embed graphics files
  \DeclareGraphicsExtensions{.pdf,.png,.jpg,.jpeg} % for pdflatex we expect .pdf, .png, or .jpg files
\else%                                 % else we use pure latex
  \ExecuteOptions{dvips}
  \usepackage{graphicx}                % allow us to embed graphics files
  \DeclareGraphicsExtensions{.eps}     % for pure latex we expect eps files
\fi%

%% it is recomended to use ``\autoref{sec:bla}'' instead of ``Fig.~\ref{sec:bla}''
\graphicspath{{figures/}{pictures/}{images/}{./}} % where to search for the images

\usepackage{microtype}                 % use micro-typography (slightly more compact, better to read)
\PassOptionsToPackage{warn}{textcomp}  % to address font issues with \textrightarrow
\usepackage{textcomp}                  % use better special symbols
\usepackage{mathptmx}                  % use matching math font
\usepackage{times}                     % we use Times as the main font
\renewcommand*\ttdefault{txtt}         % a nicer typewriter font
\usepackage{cite}
%% We encourage the use of mathptmx for consistent usage of times font
%% throughout the proceedings. However, if you encounter conflicts
%% with other math-related packages, you may want to disable it.

%% In preprint mode you may define your own headline.
%\preprinttext{To appear in IEEE Transactions on Visualization and Computer Graphics.}

%% If you are submitting a paper to a conference for review with a double
%% blind reviewing process, please replace the value ``0'' below with your
%% OnlineID. Otherwise, you may safely leave it at ``0''.
\onlineid{0}

%% declare the category of your paper, only shown in review mode
\vgtccategory{Research}

%% Paper title.
\title{Comparison of Energy Consumption in Wi-Fi and Bluetooth Communication: A Case Study on Context Aware Building}

%% This is how authors are specified in the journal style

%% indicate IEEE Member or Student Member in form indicated below
\author{Guntur Dharma Putra}
\authorfooter{
%% insert punctuation at end of each item
\item
 Guntur Dhrarma Putra is an MSc Student in Computing Science at the Universtiy of Groningen. E-mail: g.d.putra@student.rug.nl.
}

%other entries to be set up for journal
\shortauthortitle{Putra: Comparison of Energy consumption of Wi-Fi and Bluetooth Communication in Context aware Building}
%\shortauthortitle{Firstauthor \MakeLowercase{\textit{et al.}}: Paper Title}

%% Abstract section.
\abstract{
Context awareness has been an interesting topic recently. Its ability to infer whether a person exists on a particular room or building is really important for smart building. 
} % end of abstract

%% Keywords that describe your work. Will show as 'Index Terms' in journal
%% please capitalize first letter and insert punctuation after last keyword
\keywords{Context-aware, smart building, wi-fi, bluetooth low energy}

%% ACM Computing Classification System (CCS). 
%% See <http://www.acm.org/class/1998/> for details.
%% The ``\CCScat'' command takes four arguments.

% \CCScatlist{ % not used in journal version
%  \CCScat{K.6.1}{Management of Computing and Information Systems}%
% {Project and People Management}{Life Cycle};
%  \CCScat{K.7.m}{The Computing Profession}{Miscellaneous}{Ethics}
% }

%% Uncomment below to include a teaser figure.
  %  \teaser{
  %  \centering
  %  \includegraphics[width=16cm]{CypressView}
  %  \caption{In the Clouds: Vancouver from Cypress Mountain.}
  % }

%% Uncomment below to disable the manuscript note
\renewcommand{\manuscriptnotetxt}{}

%% Copyright space is enabled by default as required by guidelines.
%% It is disabled by the 'review' option or via the following command:
% \nocopyrightspace

% \vgtcinsertpkg

%%%%%%%%%%%%%%%%%%%%%%%%%%%%%%%%%%%%%%%%%%%%%%%%%%%%%%%%%%%%%%%%
%%%%%%%%%%%%%%%%%%%%%% START OF THE PAPER %%%%%%%%%%%%%%%%%%%%%%
%%%%%%%%%%%%%%%%%%%%%%%%%%%%%%%%%%%%%%%%%%%%%%%%%%%%%%%%%%%%%%%%%

\begin{document}

%% The ``\maketitle'' command must be the first command after the
%% ``\begin{document}'' command. It prepares and prints the title block.

%% the only exception to this rule is the \firstsection command
\firstsection{Introduction}

\maketitle
Smart building has been an interesting topic of research recently. One portion of research in smart home is occupancy detection, which aims to detect whether a person is present in a particular location. Its importance to detect user presence is crucial in the building energy management, since the building can manage energy allocation efficiently regarding how many persons are present.

Several methods have been proposed to overcome the occupancy detection. One of them makes use of Bluetooth Low Energy (BLE) beacon, as the beacon is useful because it always transmitting a unique data packet that indicates certain location information. Assuming that the users always bring mobile phone with them, an application can be installed on the mobile phone to scan a particular beacon, so that the application knows where currently the user is, then the application sends the data to the central server. The data can be analyzed later on to detect user presence.

Normally, the application sends the data to the server through HTTP communication done via Wi-Fi connectivity, as the BLE is already used to sense the beacon. No BLE utilization for data transmission to the server has been found. In fact, BLE is obviously more energy efficient compared to WiFi, as BLE is designed to be implemented in devices coupled with limited source of energy, e.g., battery.

This study tries to investigate BLE utilization for transmitting the occupancy data to the server. This study measures and compares the energy consumption of the mobile phone when performing data transmission via WiFif and BLE. A tailored application is developed and several possible scenario is also taken into consideration, such as number of detected sensor and user location relative to the server or access point. The result of this study may be useful for the future decision whether BLE will be implemented instead of WiFi to transmit occupancy data to the server.

%% \section{Introduction} %for journal use above \firstsection{..} instead
The rest of this report is structured as follows. Section \ref{sec:related_work} presents other related work to this study. Methodology is described in section \ref{sec:methodology}, while the results and discussion is discussed in section \ref{sec:results_and_discussion}. Lastly, a conclusion is drawn in section \ref{sec:conclusion}

\section{Related Work} % (fold)
\label{sec:related_work}
cite all the related work properly that you base your own work on

discuss why it is relevant and what is similar or different to your own work
use images (from other papers) to illustrate the related work, credit the authors with a reference in the image caption

give details for each publication (authors, title, year, page numbers, publisher, publisher address (town); for articles volume and number and month; for things other than books, articles, or papers also the type of publication)

% section related_work (end)

\section{Methodology} % (fold)
\label{sec:methodology}
As a term project, this study was performed in three consecutive months. As the main part of this study, the energy measurement was carried out in iPhone 6, which runs iOS operating system. Additionally, Asus vivo mini PC was also utilized as a thin client that runs the server application. 

\subsection{Tailored Application Development} % (fold)
\label{sub:tailored_application_development}
The application is written in Swift, which is the new general-purpose programming language developed by Apple Inc. Xcode, with Storyboard, is used to develop the application, with Git as the source code repository.

A dummy occupancy data is sent to the server, as this study imitates the real implementation of user occupancy but not necessarily involves the sensing part.

Alamofire library is used to cope the HTTP communication.
% subsection tailored_application (end)

\subsection{Measuring the Energy Consumption} % (fold)
\label{sub:tracing_}

% subsection tracing (end)

% section methodology (end)

\section{Results and Discussions} % (fold)
\label{sec:results_and_discussion}
Explain in detail first, separately for bluetooth and wifi.
Then explain the comparison of bluetooth and wifi and the baseline.
% section results_and_discussion (end)

\section{Conclusion} % (fold)
\label{sec:conclusion}
We have presented that Bluetooth is more energy efficient than Wi-Fi.

Wifi may still be a consideration because usually mobilephone are also using wifi extensively.

Some drawbacks may persist if bluetooth is implemented, such as instability. Other implementation than what is impelemented here must be looked for.
% section conclusion (end)


%% if specified like this the section will be committed in review mode
\acknowledgments{
The authors wish to thank A, B, C. This work was supported in part by
a grant from XYZ.}

%\bibliographystyle{abbrv}
\bibliographystyle{abbrv-doi}
%\bibliographystyle{abbrv-doi-narrow}
%\bibliographystyle{abbrv-doi-hyperref}
%\bibliographystyle{abbrv-doi-hyperref-narrow}
%%use following if all content of bibtex file should be shown
%\nocite{*}
\bibliography{cs-rug-internship}
\end{document}

